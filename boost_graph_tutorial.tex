%% LyX 2.0.8.1 created this file.  For more info, see http://www.lyx.org/.
%% Do not edit unless you really know what you are doing.
\documentclass[english]{article}
\usepackage[T1]{fontenc}
\usepackage[latin9]{inputenc}
\usepackage{listings}
\usepackage{float}

\makeatletter

%%%%%%%%%%%%%%%%%%%%%%%%%%%%%% LyX specific LaTeX commands.
\floatstyle{ruled}
\newfloat{algorithm}{tbp}{loa}
\providecommand{\algorithmname}{Algorithm}
\floatname{algorithm}{\protect\algorithmname}

%%%%%%%%%%%%%%%%%%%%%%%%%%%%%% User specified LaTeX commands.
\usepackage{tikz}

\makeatother

\usepackage{babel}
\begin{document}

\title{Boost.Graph tutorial}


\author{Richel Bilderbeek}

\maketitle

\section{Introduction}

I think that Boost.Graph is designed very well. Drawback is IMHO that
there are only few and even fewer complete examples using Boost.Graph. 

The book \cite{siek2001boost} is IMHO not suited best for a tutorial
as it contains heavy templated code, and an unchronological ordering
of subjects. More experienced programmers can appreciate that the
authors took great care that the code snippets written in the book
were correct: all snippets are numbered, and I'd bet they are tested
to compile. 


\subsection{Coding style used}

I prefer not to use the keyword auto, but to explicitly mention the
type instead. I think this is beneficial to beginners. When using
Boost.Graph in production code, I do prefer to use auto.


\section{Creating graphs}

Boost.Graph is about creating graphs. In this chapter we create graphs,
starting from simple to more complex. 


\subsection{Creating an empty graph}

Let's create a trivial empty graph:

\begin{algorithm}[H]
\lstinputlisting[breaklines=true,language={C++}]{create_empty_graph.cpp}

\caption{Creating an empty graph}
\end{algorithm}


Congratulations, you've just created a boost::adjacency\_list in which:
\begin{itemize}
\item The out edges are stored in a std::vector
\item The vertices are stored in a std::vector
\item The graph is directed
\item Vertices, edges and graph have no properties
\item Edges are stored in a std::list 
\end{itemize}
The boost::adjacency\_list is the most commonly used graph type, the
other is the boost::adjacency\_matrix.


\subsection{Creating $K_{2}$, a fully connected graph with two vertices}

To create a fully connected graph with two vertices (also called $K_{2}$),
one needs two vertices and one (undirected) edge, as depicted in figure
\ref{fig:k2_graph}.

\begin{figure}[H]
\tikz 
\draw[thick] 
  (0,0) node[fill=black,shape=circle,text=white] {$a$} 
    -- (5,1) node[fill=black,shape=circle,text=white] {$b$} 
;

\caption{$K_{2}$: a fully connected graph with two vertices named $a$ and
$b$\label{fig:k2_graph}}
\end{figure}


To create $K_{2}$, the following code can be used:

\lstinputlisting[breaklines=true,captionpos=b,frame=tb,language={C++}]{create_k2_graph.cpp}

Note that this code has more lines of using statements than actual
code! In this code, the third template argument of boost::adjacency\_list
is boost::undirectedS, to select (that is what the S means) for an
undirected graph. Adding a vertex with boost::add\_vertex results
in a vertex descriptor, which is a handle to the vertex added to the
graph. Two vertex descriptors are then used to add an edge to the
graph. Adding an edge using boost::add\_edge returns two things: an
edge descriptor and a boolean indicating success. In the code example,
we assume insertion is successfull.

Note that the graph lacks all properties: nodes do not have names,
nor do edges.

\bibliographystyle{plain}
\bibliography{boost_graph_tutorial}

\end{document}
